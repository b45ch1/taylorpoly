\RequirePackage{ifpdf}
\ifpdf
\documentclass[a4paper,12pt,twoside]{article}
\usepackage{microtype}
\else
\documentclass[a4paper,12pt,twoside,dvips]{article}
\fi

\usepackage[scale=0.85,includeheadfoot]{geometry}
\usepackage[medium]{titlesec}

\usepackage[utf8x]{inputenc}
\usepackage[english]{babel}

\usepackage[T1]{fontenc}

\usepackage{amsmath,amsfonts,amsthm, mathtools}
% \mathtoolsset{showonlyrefs,showmanualtags}
\mathtoolsset{showmanualtags}
\usepackage{subeqnarray}
\usepackage{latexsym}

\usepackage{mathptmx}
% \usepackage{dsfont}


\usepackage{parskip}
\usepackage{fancyhdr}
\pagestyle{fancy}
%\renewcommand{\headrulewidth}{0.4pt}


\usepackage{graphicx}
\usepackage{hyperref}

\usepackage{tikz}


\graphicspath{{pics/}}

%prevent that latex tries to fill the whole page evenly with text. This looks crap since there are often huge whitespaces between paragraphs
\raggedbottom


% ----- commands ---------

% algebraic classes
\newcommand{\R}{{\mathbb R}}
\newcommand{\K}{{\mathbb K}}
\newcommand{\N}{{\mathbb N}}
\newcommand{\M}{{\mathbb M}}

% linear algebra stuff
\newcommand{\dd}{{\rm d}}
\newcommand{\Id}{  1 \!\!\! \mathrm I}
\newcommand{\solve}{{\rm solve \;}}
\newcommand{\qr}{{\rm qr \;}}
\newcommand{\eig}{{\rm eig \;}}
\newcommand{\tr}{\mathop{\rm tr \;}}
\newcommand{\diag}{\mathop{\rm diag}}
\newcommand{\Shape}{\mathop{\rm shape}}
\newcommand{\rank}{{\mathrm{rank}\;}}


% AD stuff
\newcommand{\pb}[1]{ \overleftarrow{P}{(#1)} }
\newcommand{\pf}[1]{ \overrightarrow{P}{(#1)} }
\newcommand{\add}{\mathrm{add}\;}
\newcommand{\sub}{\mathrm{sub}\;}
\newcommand{\mul}{\mathrm{mul}\;}
% \newcommand{\div}{\mathrm{div}\;}
\newcommand{\leftshift}{{\mathrm{ls}\;}}
\newcommand{\rightshift}{{\mathrm{rs}\;}}


% DAE stuff
\newcommand{\tO}{{t_0}}
\newcommand{\tf}{ {t_f}}

% optimization stuff
\renewcommand{\L}{{\mathcal L}}
\newcommand{\argmin}{{\mathrm{argmin}\;}}


% statistical stuff
\newcommand{\E}{{\mathbb E}\;}


% abbreviations
\newcommand{\Eqn}[1]{Eqn. (\ref{#1})}
\newcommand{\Fig}[1]{Fig. (\ref{#1})}
\newcommand{\Tab}[1]{Tab. (\ref{#1})}
\newcommand{\ie} {\qtext{i.e., }}

% colors
\newcommand{\textred}[1]{#1}
\newcommand{\textblue}[1]{#1}
\newcommand{\textgreen}[1]{#1}

% chemical engineering
\newcommand{\mol}{\;[\mathrm{mol}]}
\newcommand{\kelvin}{\;[\mathrm{K}]}

% ODOE description stuff
\newcommand{\nex}{ {n_{\rm ex}}}
\newcommand{\Nex}{ {N_{\rm ex}}}
\newcommand{\Ny}{ {N_y}}
\newcommand{\ny}{ {n_y}}
\newcommand{\Nz}{ {N_z}}
\newcommand{\nz}{ {n_z}}
\newcommand{\Nu}{ {N_u}}
\renewcommand{\nu}{ {n_u}}
\newcommand{\Np}{ {N_p}}
\newcommand{\np}{ {n_p}}

\newcommand{\NF}{ {N_F}}
\newcommand{\nF}{ {n_F}}
\newcommand{\Nh}{ {N_h}}
\newcommand{\nh}{ {n_h}}
\newcommand{\Nm}{ {N_m}}
\newcommand{\nm}{ {n_m}}
\newcommand{\Nexfix}{ \Nex^{\mathrm{fix}}}
\newcommand{\Ns}{ {N_s}}
\newcommand{\ns}{ {n_s}}
\newcommand{\Nr}{ {N_r}}
\newcommand{\nr}{ {n_r}}
\newcommand{\Nv}{ {N_v}}
\newcommand{\nv}{ {n_v}}

\newcommand{\REF}{\mathrm{ref}}



% ----- environments ---------
\theoremstyle{plain}
\newtheorem{proposition}{Proposition}
\newtheorem{main}[proposition]{Main Theorem} 
\newtheorem{theorem}[proposition]{Theorem}
\newtheorem{corollary}[proposition]{Corollary} 
\newtheorem{lemma}[proposition]{Lemma}

\theoremstyle{definition}
\newtheorem{problem}{Problem}
\newtheorem{definition}[proposition]{Definition} 
\newtheorem{notation}[proposition]{Notation}
\newtheorem{algorithm}[proposition]{Algorithm} 


\theoremstyle{remark}
\newtheorem{remark}[proposition]{Remark} 
\newtheorem{bsp}[proposition]{Example}





\title{Numpy Enhancement Proposal}

\author{Sebastian~F. Walter\footnote{\texttt{sebastian.walter@gmail.com}}}



\begin{document}
\maketitle

\begin{abstract}
We propose to implement algorithms for truncated univariate Taylor polynomials over scalars (UTPS). This algebraic structure is of fundamental importance, e.g. for higher order algorithmic differentiation or Taylor series integrators for ODEs/DAEs. The algorithms are vectorized in the higher order coefficients.
\end{abstract}

A truncated univariate Taylor polynomial (UTP) $[x]_D := \sum_{d=0}^{D-1} x_d t^d,\; x \in \R$ is an element of the polynomial factor ring $R[t]/t^D$. The variable $t$ is an external parameter, i.e. the polynomial $[x]_D$ is never evaluated for a numerical $t$. The UTPs are of high importance, e.g. for use in algorithmic differentiation (AD), high order polynomial models and Taylor series integrators of ODEs/DAEs.
The basic idea is that they allow to compute higher order derivatives. As a simple example we differentiate the function $f: \R^N \rightarrow R, \; x \mapsto y = f(x)$ in the so-called \emph{forward mode of AD}:
\begin{eqnarray}
\left. \frac{\dd}{\dd t} f( x + V t) \right|_{t=0} &=& \frac{\partial f}{\partial x} V \;,
\end{eqnarray}
where $V \in R^{N \times P}$. If the gradient $\nabla f(x_0)$ is desired one can set $V$ to the identity matrix $\Id_N$ and one obtains 
\begin{eqnarray}
 \left. \frac{\dd}{\dd t} f( x + V t) \right|_{t=0} &=& \frac{\partial f}{\partial x} \Id_N = \nabla f(x)\;.
\end{eqnarray}
This generalizes to higher order derivatives by computing 
\begin{eqnarray}
\left. \frac{\dd^D}{\dd t^D} f(x + Vt) \right|_{t=0} &=& \nabla^D f(x) \{ V,\dots, V \} \;.
\end{eqnarray}
The off-diagonal elements of the derivative tensor $\nabla^D f(x)$ can be computed by a technique called \emph{exact interpolation} \cite{Griewank2008EDP}.

\paragraph{Data Structure and Algorithms}
The basic data structure is an array $[x]_{D+1}$ with coefficients
\begin{eqnarray*}
\;[x]_{D+1} &=& [x_0, x_1, \dots, x_D] \quad \quad \mbox{not vectorized} \\
\;[x]_{D+1; P} &=& [x_0, x_{11}, \dots, x_{1,D-1}, x_{21}, \dots, x_{P,D-1}] \quad \quad \mbox{vectorized} \;.
\end{eqnarray*}
I.e. the zero'th coefficient $x_0$ is the same for all $P$ directions.

To show how the algorithms have to work, we restrict to $P=1$ and look at the polynomial multiplication $z = \mul(x,y)$:
\begin{eqnarray}
[z]_{D+1} &\stackrel{D+1}{=}& [x]_{D+1} y_{D+1} \\
\sum_{d=0}^D z_d t^d &\stackrel{D+1}{=}& \left( \sum_{d=0}^D x_d t^d \right) \left( \sum_{c=0}^D y_c t^c \right)   \\
&\stackrel{D+1}{=}& \sum_{d=0}^D \sum_{k=0}^d x_k y_{d-k} t^d \\
&\stackrel{D+1}{=}& x_0 y_0 + \sum_{d=1}^D \left( \sum_{k=1}^{d-1} x_k y_{d-k} + x_0 y_d + x_d y_0 \right)  t^d \;,
\end{eqnarray}
We have used $\stackrel{D+1}{=}$ to express equality up to order $D$. That means that an (unoptimized) algorithm for the multiplication is
{\scriptsize
\begin{verbatim}
int amul(int P, int D, double *x, double *y, double *z ){
    /* 
    assignment multiplication
    computes  z += mul(x,y) in Taylor arithmetic
    */
    int k,d,p;
    double *zd, *xd, *yd;
    double *zp, *xp, *yp;

    zd = z;
    yd = y;
    xd = x;
    
    /* d = 0: base point z_0 */
    (*zd) += (*xd) * (*yd);

    for(p = 0; p < P; ++p){
        xp = x + p*(D-1);
        yp = y + p*(D-1);        
        zp = z + p*(D-1);
        
        /* d > 0: higher order coefficients */
        for(d = 1; d < D; ++d){
            zd = zp + d;
            /* compute x_0 y_d */
            yd = yp + d;
            (*zd) +=  x[0] * (*yd);
            
            /* compute sum_{k=1}^{d-1} x_k y_{d-k} */
            xd = xp + 1;
            yd = yp + d - 1;
            for(k = 1; k < d; ++k){
                (*zd) += (*xd) * (*yd);
                xd++;
                yd--;
            }
            
            xd = xp + d;
            (*zd) +=  (*xd) * y[0];        
        }
    }
    return 0;
} 
\end{verbatim}
}

 

% \tableofcontents

% \input{introduction}



\bibliographystyle{plain}
\bibliography{refs}

\end{document} 
 
